\documentclass[letterpaper, 11pt]{article}
\usepackage{latexsym}
\usepackage{amssymb}
\usepackage{times}
%\usepackage[in]{fullpage}
\usepackage{amsmath,amsfonts,amsthm}
\usepackage{graphicx}

%\documentclass[11pt]{article}
%\pagestyle{myheadings}
%\usepackage[ruled,nothing]{algorithm}
%\usepackage{algorithmic}
%\usepackage[dvips]{epsfig,graphicx}
%\numberwithin{equation}{section}

\bibliographystyle{plain}

\newenvironment{newalgo}[2]{\begin{algorithm}

\caption{\textsc{#1}}\label{#2}

\begin{algorithmic}[1]}{\end{algorithmic}\end{algorithm}}



\newcommand{\gm}{\gamma}
\newcommand{\wh}{\widehat}
\newcommand{\rep}{representation}
\newcommand{\rv}{random variable}
\newcommand{\la}{\lambda}
\newcommand{\wt}{\widetilde}
\newcommand{\st}{such that}
\newcommand{\slvary}{slowly varying}
\newcommand{\ma}{moving average}
\newcommand{\regvary}{regularly varying}
\newcommand{\asy}{asymptotic}
\newcommand{\ts}{time series}
\newcommand{\id}{infinitely divisible}
\newcommand{\seq}{sequence}
\newcommand{\fidi}{finite dimensional \ds}

\newcommand{\ble}{\begin{lemma}}
\newcommand{\ele}{\end{lemma}}
\newcommand{\bfX}{{\bf X}}
\newcommand{\pro}{probabilit}
\newcommand{\BX}{{\bf X}}
\newcommand{\BY}{{\bf Y}}
\newcommand{\BZ}{{\bf Z}}
\newcommand{\BV}{{\bf V}}
\newcommand{\BW}{{\bf W}}
\newcommand{\reals}{{\mathbb R}}
\newcommand{\bbr}{\reals}

\newcommand{\balpha}{\mbox{\boldmath$\alpha$}}
\newcommand{\bbeta}{\mbox{\boldmath$\beta$}}
\newcommand{\bmu}{\mbox{\boldmath$\mu$}}
\newcommand{\tbmu}{\mbox{\boldmath${\tilde \mu}$}}
\newcommand{\bEta}{\mbox{\boldmath$\eta$}}


\def \br#1{\left \{#1 \right \}}
\def \pr#1{\left (#1 \right)}

\newcommand{\Gm}{\Gamma}
\newcommand{\ep}{\epsilon}


\newtheorem{lemma}{Lemma}[section]
\newtheorem{figur}[lemma]{Figure}
\newtheorem{theorem}[lemma]{Theorem}
\newtheorem{proposition}[lemma]{Proposition}
\newtheorem{definition}[lemma]{Definition}
\newtheorem{corollary}[lemma]{Corollary}
\newtheorem{example}[lemma]{Example}
\newtheorem{exercise}[lemma]{Exercise}
\newtheorem{remark}[lemma]{Remark}
\newtheorem{fig}[lemma]{Figure}
\newtheorem{tab}[lemma]{Table}
\newtheorem{fact}[lemma]{Fact}
\newtheorem{test}{Lemma}
\newtheorem{algorithm}[lemma]{Algorithm}

\newcommand{\play}{\displaystyle}

\newcommand{\ms}{measure}
\newcommand{\beao}{\begin{eqnarray*}}
\newcommand{\eeao}{\end{eqnarray*}\noindent}
\newcommand{\beam}{\begin{eqnarray}}
\newcommand{\eeam}{\end{eqnarray}\noindent}

\newcommand{\halmos}{\hfill\mbox{\qed}\\}
\newcommand{\fct}{function}
\newcommand{\ins}{insurance}
\newcommand{\ds}{distribution}

\newcommand{\one}{{\bf 1}}
\newcommand{\eid}{\buildrel{\rm d}\over {=}}
\newcommand {\Or}{\rm ORDER}
\newcommand {\In}{\rm INTER}

\newcommand{\bbd}{{\mathbb D}}
\newcommand{\vi}{$V_{ij}$ }
\newcommand{\rr}{R^{\prime\prime}}
%\newcommand{\R}{R^\prime}
\newcommand{\ci}{\frac{1}{c}}
\newcommand{\Vi}{V(n)}
\newcommand{\dR}{\mathcal R}
\newcommand{\md}[1]{\left(\ \rm{mod}\ \it{#1}\right)}
\newcommand{\So}{s}
%\begin{document}
%\def\DoubleSpace{\baselineskip=24pt}
%\DoubleSpace \sloppy

\begin{document}



\title{Parallel Programming \\ Assignment \#2: Measuring Parallel Performance}
\author{Yifan Ge}


\maketitle


\section*{Problem 1}
For reasonable parameters, measure the scale-up and speed-up of this program from $1$ thread to twice as many threads as cores. Your experiment should take seconds (so that it's a significant number of trials) but not hours. For example, for $8$ cores, you might run the program for $1,2,3,\dots,16$ threads at $1000000000$ coin tosses for speedup. (Don't worry about +/- rounding errors for number of flips per thread.) And, run the program for $1,2,3,\dots, 16$ threads at
$1000000000,2000000000,$\\$\dots,16000000000$ for scaleup. \\

\textbf{Scaleup and speedup}\\
\begin{enumerate}
    \item \textbf{Produce charts that show the scaleup and speedup of your program.}\\
        The speedup of the program is shown in Figure \ref{fig:coinspeedup}, and the scaleup is shown in Figure \ref{fig:coinscaleup}.

        \begin{figure}[h]
            \centering
            \includegraphics[width=3.8in]{coinspeedup.eps}
            \caption{Speedup of CoinFlip for different number of threads}
            \label{fig:coinspeedup}
        \end{figure}

        \begin{figure}[h]
            \centering
            \includegraphics[width=3.8in]{coinscaleup.eps}
            \caption{Scaleup of CoinFlip for different number of threads}
            \label{fig:coinscaleup}
        \end{figure}
    \item \textbf{Algorithm (true) speedup/scaleup measures the scaling performance of the algorithm as a function of processing elements. In this case, from $1\dots8$. Characterize the algorithmic speedup/scaleup. If it is sub-linear, describe the potential sources of loss.}\\
        When the number of threads ranging from $1\dots8$, both the speedup and scaleup are almost linear. Only when the number of threads reaches $8$, the speedup/scaleup start to degrade. This is because I am running the program on a $8$-core machine, so the optimal number of threads is $8$. And the JVM is running a master thread, so the speedup and scaleup cannot keep linear when the number of slave threads reaches $8$.

    \item Why does the speedup not continue to increase past the number of cores? Does it degrade? Why?  \\
        As the processor has $8$ cores, it can have at most $8$ instructions executed per cycle. When I have $8$ threads running, the processor's execution resources have been well utilized. This is why the speedup does not continue to inscrease past the number of core. An obvious degradation can be observed when there are $9$ threads running. This is because of the latency of thread switching and startup cost.

\end{enumerate} 

\textbf{Design and run an experiment that measures the startup costs of this code.}
\begin{enumerate}
    \item Describe your experiment. Why does it measure startup?
    \item Estimate startup cost. Justify your answer.
    \item Assuming that the startup costs are the serial portion of the code and the remaining time is the parallel portion of the code,what speedup would you expect to realize on 100 threads? 500 threads? 1000 threads? (Use Amdahl's law.)
\end{enumerate}
\section*{Problem 2}


\end{document}




































